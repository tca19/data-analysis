\documentclass{beamer}
\usepackage{graphicx}
\usepackage{multicol}
\usepackage{amsmath}
\usepackage{booktabs}

\graphicspath{{latex/}{img/}}

% for custom colors
\definecolor{maroon}{RGB}{215,0,0}

\mode<presentation>
{
    % theme + colorscheme
    \usetheme{Madrid}
    \usecolortheme{beaver}

    % customize bullets in items list
    \setbeamertemplate{itemize item}{\color{red}$\bullet$}
    \setbeamertemplate{itemize subitem}{\color{red}$\circ$}

    % customize bullets in table of content
    \setbeamerfont{section number projected}{%
      family=\sffamily,series=\bfseries,size=\normalsize}
    \setbeamercolor{section number projected}{bg=maroon, fg=white}
    \setbeamercolor{subsection number projected}{bg=maroon}
    \setbeamertemplate{section in toc}[circle]
    \setbeamertemplate{subsection in toc}[square]

    % remove navigation symbols
    \setbeamertemplate{navigation symbols}{}
}

% customize footline: print section/subsection/date+slides_number instead of
% author/short_title/date+slides_number.
% You might need to remove it if you use another theme than Madrid.
\makeatletter
\setbeamertemplate{footline}
{
  \leavevmode%
  \hbox{%
  \begin{beamercolorbox}[wd=.33333\paperwidth,ht=2.25ex,dp=1ex,center]
    {section in head/foot}\usebeamerfont{section in head/foot}\insertsection
  \end{beamercolorbox}% if no % at the end, there is a gap between subboxes
  \begin{beamercolorbox}[wd=.33333\paperwidth,ht=2.25ex,dp=1ex,center]
    {title in head/foot}\usebeamerfont{title in head/foot}\insertsubsection
  \end{beamercolorbox}%
  \begin{beamercolorbox}[wd=.33333\paperwidth,ht=2.25ex,dp=1ex,right]
    {date in head/foot}\usebeamerfont{date in head/foot}\insertshortdate{}
    \hspace*{2em} \insertframenumber{} / \inserttotalframenumber\hspace*{2ex}
  \end{beamercolorbox}}%
  \vskip0pt
}
\makeatother

% main information
\title[Algorithms for Data Analysis]{Algorithms for Data Analysis}
\author{Julien Tissier}
\date{\today}

\begin{document}

%%%%%%%%%%%%%%%%%%%%%%%
%%    F R A M E S    %%
%%%%%%%%%%%%%%%%%%%%%%%


% section -> on the left in footline; subsection -> in the center footline
\section*{\insertshortauthor} % asterisk -> do not appear in table of content
\subsection*{\insertshorttitle}

% frame with title + main information
\begin{frame}
    \titlepage
\end{frame}

% summary frame
\begin{frame}
  \frametitle{Overview}
%  \begin{multicols}{2}
    \tableofcontents
%  \end{multicols}
\end{frame}

%------------------
%    I N T R O    -
%------------------

\section{Introduction to Data Analysis}

% data (analysis) definition
\subsection{Definition}
\begin{frame}
  \frametitle{What is data analysis?}
  \begin{block}{Data}
    Pieces of information (measurement, values, facts...) that can be :
    \begin{itemize}
      \item structured (matrices, tabular data, RDBMS, time series...)
      \item unstructured (news articles, webpages, images/video...)
    \end{itemize}
  \end{block}
  \begin{block}{Data Analysis}
  Process of preparing, transforming and using models to find more information
  from data, as well as visualizing results.
  \end{block}
\end{frame}

% tools (why Python?)
\subsection{Tools and libraries}
\begin{frame}
  \frametitle{How to analyze data?}
  There are usually two steps in data analysis. The first one is to find and
  \textbf{develop models} that can extract useful information from data (with
  languages like R or MATLAB). The second one is to \textbf{develop programs}
  that can be used in production systems (with languages like Java or C++).
  \\~\\

  With a growing popularity among scientists as well as the development of
  efficient libraries (numpy, pandas), \textbf{Python} became a great tool for
  data analysis. Python has many advantages :
  \begin{itemize}
    \item great for string/data processing
    \item can be used for both prototyping/production
    \item has a lot of existing libraries
    \item can easily integrate C/C++/FORTRAN legacy code
    \item easy to read/develop
  \end{itemize}
\end{frame}

% list of libraries
\begin{frame}
  \frametitle{Libraries}
  This course will be based on Python 3.7 (or above) and the following
  libraries :
  \begin{itemize}
    \item IPython (7.8+) : enhanced Python shell
    \item numpy (1.17+) : fast/efficient arrays and operations
    \item pandas (0.25+) : data structures (Series/DataFrame)
    \item matplotlib (3.1+) : plots and 2D visualization
    \item scipy (1.3.1+) : scientific algorithms
    \item scikit-learn (0.21+) : machine learning algorithms
  \end{itemize}

  Jupyter Notebook will also be used to give you samples of code, as they
  provide a more interactive way to learn and discover how these libraries
  work.
\end{frame}

% numpy
\subsection{Numpy}
\begin{frame}
  \frametitle{Numpy}
  Numpy (\textbf{Num}erical \textbf{Py}thon) is a high performance scientific
  computing library that can be used for matrices computations, Fourier
  transforms, linear algebra, statistical computations...
  \\~\\

  The main type of data in Numpy is the \textbf{ndarray} :
  \begin{itemize}
    \item n-dimensional array
    \item fixed size
    \item homogeneous datatypes
    \item similar to C arrays (continuous block of memory)
  \end{itemize}
\end{frame}

\begin{frame}
  \frametitle{Why is Numpy efficient?}
  As a high level language, Python is slow to do any heavy computations,
  especially if very large arrays are involved. Numpy solves this problem thanks
  to the ndarray datatypes :
  \begin{itemize}
    \item efficient memory management (continuous block)
    \item use C loops instead of Python loops for computations on array
    \item vectorized operations (computations are done block by block, not
      element by element)
    \item rely on low-level routines for some operations (BLAS/LAPACK)
  \end{itemize}
\end{frame}

\begin{frame}[fragile]
  \frametitle{Numpy}
  \begin{example}
    \begin{verbatim}

  import numpy as np

  a = np.array([1, 2, 3, 4])
  b = np.array([6, 7, 8, 9])

  c = a * b

  d = np.array([[1, 2], [3, 4]])
    \end{verbatim}
  \end{example}
\end{frame}

% pandas
\subsection{Pandas}
\begin{frame}
  \frametitle{Pandas}
  Pandas is a high-performance Python library used to work with data and analyze
  them. It contains many pre-implemented methods to read and parse data, as well
  as common statistical computations (mean, variance, correlation...).
  \\~\\

  Pandas has two main datatypes:
  \begin{itemize}
    \item Series : one-dimensional container. Indexes can be integers (like an
      array) or other objects (string, date...)
    \item DataFrame : tabular data, like a spreadsheet. It contains multiple
      rows and multiple columns. It can be seen as a collection of Series.
  \end{itemize}
\end{frame}

\begin{frame}[fragile]
  \frametitle{Pandas}
  \begin{example}[Series]
    \begin{verbatim}

  from pandas import Series

  s1 = Series([4, 7, 9, -1])
  s2 = Series([12, 3.2, "John"],
              index=["mark", "gpa", "name"])
    \end{verbatim}
  \end{example}

  \begin{example}[DataFrame]
    \begin{verbatim}

  from pandas import DataFrame
  df = DataFrame({"key1": [1, 2, 3],
                  "key2": [4, 5, 6]})
    \end{verbatim}
  \end{example}
\end{frame}

% matplotlib
\subsection{Matplotlib}
\begin{frame}
  \frametitle{Matplotlib}
  Matplotlib is a plotting library used to visualize data and create graphics.
  Pandas directly uses matplotlib for representation.
  \begin{figure}[h]
    \includegraphics[width=8cm]{img/matplotlib_1.png}
  \end{figure}
\end{frame}

\begin{frame}
  \frametitle{Matplotlib}
  Matplotlib is a plotting library used to visualize data and create graphics.
  Pandas directly uses matplotlib for representation.
  \begin{figure}[h]
    \includegraphics[width=8cm]{img/matplotlib_2.png}
  \end{figure}
\end{frame}
\end{document}
